\section{Webserver}

\subsection{Web API}

\begin{table}[H]
	\rowcolors{2}{gray!25}{white}
	\centering
	\caption{API addresser for User}
	\label{tab:web_user}
	\begin{tabular}{|L{0.3\textwidth}|L{0.70\textwidth}|}
		\hline
		\rowcolor{gray!50}
		\multicolumn{2}{|l|}{Users}\\
		\hline
		GET \newline
		/api/Users &
		Henter brugeren med det givne id, er id'et sat til -1 vil den bruger der er logget ind med, returneres. \\
        \hline
        GET \newline
		/api/Users/\{id\} &
		Henter brugeren med det givne id, er id'et sat til -1 vil den bruger der er logget ind med, returneres. \\
        \hline
        GET \newline
		/api/Users/email/\{email\} &
		Henter brugeren med det givne id, er id'et sat til -1 vil den bruger der er logget ind med, returneres. \\
        \hline
		PUT \newline
		/api/Users/\{id\} &
		Opdaterer brugerinformationer omkring den bruger, som er logget ind. \\
		\hline
		DELETE \newline
		/api/Users/\{id\} &
		Sletter den bruger, som er logget ind. \\
		\hline
		POST \newline
		/api/Users &
		Opretter en ny bruger. \\
		\hline
	\end{tabular}
\end{table}

\begin{table}[H]
	\rowcolors{2}{gray!25}{white}
	\centering
	\caption{API addresser for Authentication}
	\label{tab:web_user}
	\begin{tabular}{|L{0.3\textwidth}|L{0.70\textwidth}|}
		\hline
		\rowcolor{gray!50}
		\multicolumn{2}{|l|}{Authentication}\\
		\hline
		GET \newline
		/api/Authentication/\{id\} &
		Henter brugeren med det givne id, er id'et sat til -1 vil den bruger der er logget ind med, returneres. \\
		\hline
		POST \newline
		/api/Authentication/authenticate &
		Opretter en ny bruger. \\
        \hline
        POST \newline
		/api/Authentication/register &
		Opretter en ny bruger. \\
        \hline
        PUT \newline
		/api/Authentication/\{id\} &
		Opretter en ny bruger. \\
		\hline
		DELETE \newline
		/api/Authentication/\{id\} &
		POST bærer anmodningsparameter, hvilket er en mere sikker måde at overføre data fra klient til server i http-protokol på. \\
		\hline
	\end{tabular}
\end{table}

\begin{table}[H]
	\rowcolors{2}{gray!25}{white}
	\centering
	\caption{API addresser for Audits}
	\label{tab:web_user}
	\begin{tabular}{|L{0.3\textwidth}|L{0.70\textwidth}|}
		\hline
		\rowcolor{gray!50}
		\multicolumn{2}{|l|}{Audits}\\
		\hline
		GET \newline
		/api/Audits &
		Henter brugeren med det givne id, er id'et sat til -1 vil den bruger der er logget ind med, returneres. \\
        \hline
        GET \newline
		/api/Audits/\{id\} &
		Sender et ønske om at få tilsendt en email til at få ændret sit kodeord, uden at være logget ind. 
		\\
        \hline
        GET \newline
		/api/Audits/\{id\} &
		Sender et ønske om at få tilsendt en email til at få ændret sit kodeord, uden at være logget ind. 
		\\
		\hline
		PUT \newline
		/api/Audits/email/\{email\} &
		Opdaterer brugerinformationer omkring den bruger, som er logget ind. \\
		\hline
		DELETE \newline
		/api/Audits/\{id\} &
		Sletter den bruger, som er logget ind. \\
		\hline
		POST \newline
		/api/Audits &
		Opretter en ny bruger. \\
		\hline
		
	\end{tabular}
\end{table}

\begin{table}[H]
	\rowcolors{2}{gray!25}{white}
	\centering
	\caption{API addresser for Biddings}
	\label{tab:web_user}
	\begin{tabular}{|L{0.3\textwidth}|L{0.70\textwidth}|}
		\hline
		\rowcolor{gray!50}
		\multicolumn{2}{|l|}{Biddings}\\
		\hline
		GET \newline
		/api/Biddings &
		Henter brugerens bud på et projekt. \\
        \hline
        GET \newline
		/api/Biddings/freelancer/\{id\} &
		Henter brugerens bud på et projekt. \\
        \hline
        GET \newline
		/api/Biddings/project/\{projectId\} &
		Henter brugerens bud på et projekt. \\
        \hline
        GET \newline
		/api/Biddings/\{id\} &
		Henter brugerens bud på et projekt. \\
		\hline
		PUT \newline
		/api/Biddings/\{bidId\} &
		Opdaterer det bud som brugeren har indtastet til et projekt. \\
		\hline
		DELETE \newline
		/api/Biddings/\{id\} &
		Brugeren kan slette sit bud. \\
		\hline
		POST \newline
		/api/Biddings &
		Brugeren kan oprette et ny bud. \\
		\hline
	
	\end{tabular}
\end{table}

\begin{table}[H]
	\rowcolors{2}{gray!25}{white}
	\centering
	\caption{API addresser for Collaborations}
	\label{tab:web_user}
	\begin{tabular}{|L{0.3\textwidth}|L{0.70\textwidth}|}
		\hline
		\rowcolor{gray!50}
		\multicolumn{2}{|l|}{Collaborations}\\
		\hline
		GET \newline
		/api/Collaborations &
		Henter status på et projekt . \\
		\hline
        GET \newline
		/api/Collaborations/project/{projectId} &
		Ændrer kodeordet for den bruger, som er logget ind. \\
		\hline
		GET \newline
		/api/Collaborations/{id}/\{email\} &
		Sender et ønske om at få tilsendt en email til at få ændret sit kodeord, uden at være logget ind. 
		\\
		\hline
		POST \newline
		/api/Collaborations &
		Opretter en ny bruger. \\
		\hline
	
	\end{tabular}
\end{table}

\begin{table}[H]
	\rowcolors{2}{gray!25}{white}
	\centering
	\caption{API addresser for Files}
	\label{tab:web_user}
	\begin{tabular}{|L{0.3\textwidth}|L{0.70\textwidth}|}
		\hline
		\rowcolor{gray!50}
		\multicolumn{2}{|l|}{Files}\\
		\hline
		GET \newline
		/api/Profiles &
		Henter brugeren med det givne id, er id'et sat til -1 vil den bruger der er logget ind med, returneres. \\
        \hline
        GET \newline
		/api/Profiles/\{id\} &
		Henter brugeren med det givne id, er id'et sat til -1 vil den bruger der er logget ind med, returneres. \\
        \hline
        GET \newline
		/api/Profiles/user/\{userId\} &
		Henter brugeren med det givne id, er id'et sat til -1 vil den bruger der er logget ind med, returneres. \\
		\hline
		PUT \newline
		/api/Profiles/\{profileId\} &
		Opdaterer brugerinformationer omkring den bruger, som er logget ind. \\
		\hline
		DELETE \newline
		/api/Profiles/\{id\} &
		Sletter den bruger, som er logget ind. \\
		\hline
		POST \newline
		/api/Profiles &
		Opretter en ny bruger. \\
		\hline
	
	\end{tabular}
\end{table}

\begin{table}[H]
	\rowcolors{2}{gray!25}{white}
	\centering
	\caption{API addresser for Payments}
	\label{tab:web_user}
	\begin{tabular}{|L{0.3\textwidth}|L{0.70\textwidth}|}
		\hline
		\rowcolor{gray!50}
		\multicolumn{2}{|l|}{Payments}\\
		\hline
		POST \newline
		/api/Payments/deposit &
		Sender et ønske om at få tilsendt en email til at få ændret sit kodeord, uden at være logget ind. 
		\\
		\hline
	\end{tabular}
\end{table}

\begin{table}[H]
	\rowcolors{2}{gray!25}{white}
	\centering
	\caption{API addresser for Profiles}
	\label{tab:web_user}
	\begin{tabular}{|L{0.3\textwidth}|L{0.70\textwidth}|}
		\hline
		\rowcolor{gray!50}
		\multicolumn{2}{|l|}{Profiles}\\
		\hline
		GET \newline
		/api/Profiles &
		Henter brugeren med det givne id, er id'et sat til -1 vil den bruger der er logget ind med, returneres. \\
        \hline
        GET \newline
		/api/Profiles/\{id\} &
		Henter brugeren med det givne id, er id'et sat til -1 vil den bruger der er logget ind med, returneres. \\
        \hline
        GET \newline
		/api/Profiles/user/\{userId\} &
		Henter brugeren med det givne id, er id'et sat til -1 vil den bruger der er logget ind med, returneres. \\
		\hline
		PUT \newline
		/api/Profiles/\{profileId\} &
		Opdaterer brugerinformationer omkring den bruger, som er logget ind. \\
		\hline
		DELETE \newline
		/api/Profiles/\{id\} &
		Sletter den bruger, som er logget ind. \\
		\hline
		POST \newline
		/api/Profiles &
		Opretter en ny bruger. \\
		\hline
		
	\end{tabular}
\end{table}

\begin{table}[H]
	\rowcolors{2}{gray!25}{white}
	\centering
	\caption{API addresser for Projects}
	\label{tab:web_user}
	\begin{tabular}{|L{0.3\textwidth}|L{0.70\textwidth}|}
		\hline
		\rowcolor{gray!50}
		\multicolumn{2}{|l|}{Projects}\\
		\hline
		GET \newline
		/api/Projects &
		Henter brugeren med det givne id, er id'et sat til -1 vil den bruger der er logget ind med, returneres. \\
        \hline
        GET \newline
		/api/Projects/employer/\{id\} &
		Henter brugeren med det givne id, er id'et sat til -1 vil den bruger der er logget ind med, returneres. \\
        \hline
        GET \newline
		/api/Projects/freelancer/\{id\} &
		Henter brugeren med det givne id, er id'et sat til -1 vil den bruger der er logget ind med, returneres. \\
        \hline
        GET \newline
		/api/Projects/user/\{userId\} &
		Henter brugeren med det givne id, er id'et sat til -1 vil den bruger der er logget ind med, returneres. \\
        \hline
        GET \newline
		/api/Projects/open &
		Henter brugeren med det givne id, er id'et sat til -1 vil den bruger der er logget ind med, returneres. \\
        \hline
        GET \newline
		/api/Projects/\{id\} &
		Henter brugeren med det givne id, er id'et sat til -1 vil den bruger der er logget ind med, returneres. \\
		\hline
		PUT \newline
		/api/Projects/\{id\} &
		Opdaterer brugerinformationer omkring den bruger, som er logget ind. \\
		\hline
		DELETE \newline
		/api/Projects/\{id\} &
		Sletter den bruger, som er logget ind. \\
		\hline
		POST \newline
		/api/Projects &
		Opretter en ny bruger. \\
		\hline
	\end{tabular}
\end{table}

\begin{table}[H]
	\rowcolors{2}{gray!25}{white}
	\centering
	\caption{API addresser for Chat}
	\label{tab:web_user}
	\begin{tabular}{|L{0.3\textwidth}|L{0.70\textwidth}|}
		\hline
		\rowcolor{gray!50}
		\multicolumn{2}{|l|}{Chat}\\
		\hline
		GET \newline
		/api/User/\{id\} &
		Henter brugeren med det givne id, er id'et sat til -1 vil den bruger der er logget ind med, returneres. \\
		\hline
		PUT \newline
		/api/User/\{id\} &
		Opdaterer brugerinformationer omkring den bruger, som er logget ind. \\
		\hline
		DELETE \newline
		/api/User/\{id\} &
		Sletter den bruger, som er logget ind. \\
		\hline
		POST \newline
		/api/User/ &
		Opretter en ny bruger. \\
		\hline
	
	\end{tabular}
\end{table}

\begin{table}[H]
	\rowcolors{2}{gray!25}{white}
	\centering
	\caption{API addresser for Video chat}
	\label{tab:web_user}
	\begin{tabular}{|L{0.3\textwidth}|L{0.70\textwidth}|}
		\hline
		\rowcolor{gray!50}
		\multicolumn{2}{|l|}{Video chat}\\
		\hline
		GET \newline
		/api/User/\{id\} &
		Henter brugeren med det givne id, er id'et sat til -1 vil den bruger der er logget ind med, returneres. \\
		\hline
		PUT \newline
		/api/User/\{id\} &
		Opdaterer brugerinformationer omkring den bruger, som er logget ind. \\
		\hline
		DELETE \newline
		/api/User/\{id\} &
		Sletter den bruger, som er logget ind. \\
		\hline
		POST \newline
		/api/User/ &
		Opretter en ny bruger. \\
		\hline
	\end{tabular}
\end{table}